% LaTeX Curriculum Vitae Template
%
% Copyright (C) 2004-2009 Jason Blevins <jrblevin@sdf.lonestar.org>
% http://jblevins.org/projects/cv-template/
%
% You may use use this document as a template to create your own CV
% and you may redistribute the source code freely. No attribution is
% required in any resulting documents. I do ask that you please leave
% this notice and the above URL in the source code if you choose to
% redistribute this file.

\documentclass[letterpaper]{article}

\usepackage{hyperref}
\usepackage{enumitem}
\usepackage{geometry}

% Comment the following lines to use the default Computer Modern font
% instead of the Palatino font provided by the mathpazo package.
% Remove the 'osf' bit if you don't like the old style figures.
\usepackage[T1]{fontenc}
\usepackage[sc,osf]{mathpazo}

% Set your name here
\def\name{Siddhartha Banerjee}

% Replace this with a link to your CV if you like, or set it empty
% (as in \def\footerlink{}) to remove the link in the footer:
\def\footerlink{}

% The following metadata will show up in the PDF properties
\hypersetup{
  colorlinks = true,
  urlcolor = black,
  pdfauthor = {\name},
  pdfkeywords = {robotics, human-robot interaction, artificial intelligence},
  pdftitle = {\name: Curriculum Vitae},
  pdfsubject = {Curriculum Vitae},
  pdfpagemode = UseNone
}

\geometry{
  body={6.5in, 8.5in},
  left=1.0in,
  top=1.25in
}

% Customize page headers
\pagestyle{myheadings}
\markright{\name}
\thispagestyle{empty}

% Custom section fonts
\usepackage{sectsty}
\sectionfont{\rmfamily\mdseries\Large}
\subsectionfont{\rmfamily\mdseries\itshape\large}

% Other possible font commands include:
% \ttfamily for teletype,
% \sffamily for sans serif,
% \bfseries for bold,
% \scshape for small caps,
% \normalsize, \large, \Large, \LARGE sizes.

% Don't indent paragraphs.
\setlength\parindent{0em}

% Make lists without bullets
\renewenvironment{itemize}{
  \begin{list}{}{
    \setlength{\leftmargin}{1.5em}
  }
}{
  \end{list}
}

\begin{document}

% Place name at left
{\huge \name}

% Alternatively, print name centered and bold:
%\centerline{\huge \bf \name}

\vspace{0.25in}

\begin{minipage}{0.45\linewidth}
  \href{}{Georgia Institute of Technology} \\
  RAIL Lab, College of Computing \\
  800 Atlantic Dr NW \\
  Atlanta, GA 30332
\end{minipage}
\begin{minipage}{0.45\linewidth}
  \begin{tabular}{ll}
    Phone: & +1 (203) 435-1923 \\
    Email: & \href{mailto:siddhartha.banerjee@gatech.edu}{\tt siddhartha.banerjee@gatech.edu} \\
    Homepage: & \href{http://www.banerjs.com/}{\tt http://www.banerjs.com/} \\
  \end{tabular}
\end{minipage}

\section*{Education}

\begin{description}[leftmargin=7.5em, style=nextline]
  \item[2015 --- ] Ph.D. in Robotics. Advised by Dr. Sonia Chernova. \textit{In Progress}. \\ \textit{Georgia Institute of Technology, Atlanta, GA, USA}.

  \item[2009 --- 2013] B.S. Electrical Engineering/Computer Science with Distinction. \\ \textit{Yale University, New Haven, CT, USA}.
\end{description}

\section*{Employment}

\begin{description}[leftmargin=7.5em, style=nextline]
  \item[Fall 2018] Teaching Assistant for CS7633: Human-Robot Interaction. \\
\textit{Georgia Institute of Technology, Atlanta, GA, USA}

  \item[Summer 2018] Robotics Software Engineering Intern. Advised by Vivian Chu. \\
\textit{Diligent Robotics, Austin, TX, USA}

  \item[Summer 2017] Research Intern. Advised by Dan Bohus and Sean Andrist. \\ \textit{Microsoft, Seattle, WA, USA}

  \item[Fall 2016] Teaching Assistant for CS6601: Introduction to Artificial Intelligence. \\ \textit{Georgia Institute of Technology, Atlanta, GA, USA}

  \item[2013 --- 2015] Software Engineer on Data Team. \\ \textit{Redfin, Seattle, WA, USA}

  \item[2012 --- 2013] Peer Tutor for CPSC 202: Mathematical Tools for Computer Science. \\ \textit{Yale University, New Haven, CT, USA}

  \item[Summer 2012] Hardware Verification Intern. \\ \textit{Microsoft, Mountain View, CA, USA}
\end{description}


\section*{Publications}

\subsection*{Journals}

\begin{itemize}

  \item \textbf{S. Banerjee}, A. Silva, and S. Chernova, ``Robot Classification of Human Interruptibility and a Study of Its Effects,'' in \textit{ACM Transactions on Human-Robot Interaction (THRI)}, 7(2), p. 14, 2018

\end{itemize}

\subsection*{Conference Proceedings}

\begin{itemize}
  \item D. Kent, \textbf{S. Banerjee}, and S. Chernova, ``Learning Sequential Decision Tasks for Robot Manipulation with Abstract Markov Decision Processes and Demonstration-Guided Exploration,'' in \textit{18th Int. Conf. on Humanoid Robots}, IEEE-RAS, 2018

  \item \textbf{S. Banerjee} and S. Chernova, ``Temporal Models for Robot Classification of Human Interruptibility,'' in \textit{Int. Conf. on Autonomous Agents \& Multiagent Systems}, no. 16. IFAAMAS, 2017, pp. 1350--1359
\end{itemize}

\subsection*{Workshops}

\begin{itemize}
  \item \textbf{S. Banerjee} and S. Chernova, ``Efficient Human-Robot Interaction for Robust Autonomy in Task Execution,'' in \textit{ACM/IEEE Int. Conf. on Human-Robot Interaction Pioneers Workshop}, 2018

  \item A. Silva, \textbf{S. Banerjee}, and S. Chernova, ``Excuse Me, Could You Please Assemble These Blocks For Me?'' in \textit{What Could Go Wrong? Workshop at HRI}, 2018

  \item \textbf{S. Banerjee} and S. Chernova, ``Robots Predicting the Interruptibility of Humans,'' in \textit{RSS Workshop on Planning for HRI}, 2016

  \item B. Harrison, \textbf{S. Banerjee}, and M. O. Riedl, ``Learning from Stories: Using Natural Communication to Train Believable Agents,'' in \textit{IJCAI Workshop on Interactive Machine Learning}, 2016
\end{itemize}

\section*{Awards and Leadership Positions}

\begin{description}[leftmargin=7.5em, style=nextline]
\itemsep0em
  \item[HRI 2019] Panel Chair, \textit{HRI Pioneers Workshop}

  \item[2018 --- 2019] Co-chair, Student Activities Committee, \textit{IEEE Robotics and Automation Society}

  \item[2017 --- 2018] President, RoboGrads, \textit{Georgia Institute of Technology}

  \item[2016 --- 2017] Social Chair, RoboGrads, \textit{Georgia Institute of Technology}

  \item[Q3 2014] Employee of the Quarter, \textit{Redfin}

  \item[2012 --- 2013] Team Mentor, Formula Hybrid FSAE Team, \textit{Yale University}

  \item[2011 --- 2012] Vice President, Formula Hybrid FSAE Team, \textit{Yale University}

  \item[Summer 2011] Yale Entrepreneurial Institute Fellowship, \textit{Yale University}
\end{description}

\section*{Projects}

\begin{description}[leftmargin=7.5em, style=nextline]
  \item[Spring 2016] \textbf{Quadrotor Control via Backstepping}. \textit{Class Project} \\
  Verified and simulated the control of a quadrotor through Backstepping to show provably correct control that uses less energy than traditional Inner-loop Outer-loop control.

  \item[Spring 2016] \textbf{Treeminder: An SMS-based Goal Completion System for the United Way Achievement Club}. \textit{Class Project} \\
  Designed a goal tracking and completion system in partnership with the United Way Achievement Club to help members of at-risk populations avoid homelessness. Conducted usability and feasibility analyses to justify and support the design.

  \item[2009 --- 2013] \textbf{Yale Formula Hybrid FSAE Team}. \textit{Student Organization} \\
  Designed and built formula style gas-electric hybrid car to compete against other schools in an annual national competition. Team awards: Best Hybrid Car (2013), Ford Efficiency Award (2013), Chrysler Innovation Award (2013), GM Best Engineered Hybrid System Award (2010, 2013)

  \item[2012 --- 2013] \textbf{Synchronization and Collective Behaviour}. \textit{Senior Class Project} \\
  Simulated agent-based modeling of multi-agent systems. Explored the role of synchronization and chaos in dynamical systems.

  \item[Spring 2012] \textbf{Assigning Blame to Self-Driving Cars}. \textit{Class Project} \\
  Surveyed drivers to determine whether blame is assigned to a self-driving car or the human driver using simulations of accidents between self-driving cars with human-driven cars.

  \item[Fall 2011] \textbf{Design and Fabrication of Simple Data Encryption Standard (S-DES) Encryption/Decryption chip}. \textit{Class Project} \\
  Designed a VLSI chip to perform S-DES encryption/decryption and created CAD models of the chip and its layout in preparation for fabrication. Tested and verified the function of the chip post-fabrication.
\end{description}

\section*{Professional Memberships}

\begin{itemize}
  \item Institute for Electrical and Electronics Engineers (IEEE)
\end{itemize}

\bigskip

% Footer
\begin{center}
  \begin{footnotesize}
    Last updated: \today \\
    \href{\footerlink}{\texttt{\footerlink}}
  \end{footnotesize}
\end{center}

\end{document}
